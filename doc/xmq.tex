\documentclass[10pt,a4paper]{article}
\usepackage{rail}
\usepackage{parskip}
\usepackage{changepage}
\usepackage{geometry}

\makeatletter
\newcommand*{\shifttext}[2]{%
  \settowidth{\@tempdima}{#2}%
  \raisebox{0pt}[0pt][0pt]{%
  \makebox[\@tempdima]{\hspace*{#1}#2}}%
}
\makeatother

\geometry{a4paper, layoutwidth=20cm, layoutheight=30cm}

\newcommand{\shiftleft}[2]{\makebox[0pt][r]{\makebox[#1][l]{#2}}}
\newcommand{\shiftright}[2]{\makebox[#1][r]{\makebox[0pt][l]{#2}}}

\railoptions{-t}
\relax

\pagestyle{empty}

\railalias{LBRACE}{\{}
\railalias{RBRACE}{\}}
\railalias{BA}{\textbackslash}

\railterm{BA}
\railterm{LBRACE}
\railterm{RBRACE}

\begin{document}

\texttt{Specification for XMQ by Fredrik Öhrström 2021-01-19 \texttt{oehrstroem@gmail.com} \hfill 1/3}

\vspace{2mm}

\texttt{An xmq file must be valid UTF8 and any CRLF is changed into LF before lexing/parsing.}

\vspace{2mm}

\texttt{Lexer}

\vspace{2mm}

\raisebox{32pt}{WS:}
\begin{minipage}{15cm}
\begin{rail}
'0x20 space' | '0x09 tab' | '0x0a lf' | '0x0d cr'
\end{rail}
\end{minipage}

\raisebox{55pt}{TOKS:}
\begin{minipage}{15cm}
\begin{rail}
"'" | "=" | "(" | ")" | LBRACE | RBRACE
\end{rail}
\end{minipage}

\raisebox{-4pt}{TEXT:}
\begin{minipage}{15cm}
\begin{rail}
"UTF8 excluding WS and TOKS and NULL"
\end{rail}
\end{minipage}

\raisebox{32pt}{QUOTE:}
\begin{minipage}{15cm}
\begin{rail}
  "''"
  |"'" "UTF8 with no '" "'"
  | "'''" "UTF8 max 2 consec '" "'''"
  | "'"[$\times$ n] "UTF8 max n-1 consec '" "'"[$\times$ n]
\end{rail}
\end{minipage}

\raisebox{20pt}{QUOTES:}
\begin{minipage}{15cm}
  \begin{rail}
  QUOTE + (BA 'n'? '0x0a' WS)
\end{rail}
\end{minipage}

\raisebox{8pt}{COMMENT:}
\begin{minipage}{15cm}
\begin{rail}
  '//' UTF8 "0x0a"
  | '/*' UTF8 '*/'
\end{rail}
\end{minipage}

\pagebreak

\texttt{Parser \hfill 2/3}

\vspace{2mm}

\raisebox{8pt}{xmq:}
\begin{minipage}{15cm}
\begin{rail}
  doctype? nodes
\end{rail}
\end{minipage}

\raisebox{45pt}{nodes:}
\begin{minipage}{15cm}
\begin{rail}
  (node | QUOTES[standalone] | COMMENT | pi) +
\end{rail}
\end{minipage}

\raisebox{33pt}{node:}
\begin{minipage}{15cm}
\begin{rail}
  TEXT[tag] attributes? ( | ('=' ( TEXT | QUOTES) | LBRACE nodes RBRACE))
\end{rail}
\end{minipage}

\raisebox{32pt}{attributes:}
\begin{minipage}{15cm}
\begin{rail}
'(' (TEXT[attribute] ( | '=' ( TEXT | QUOTES ) ) +) ')'
\end{rail}
\end{minipage}

\raisebox{20pt}{pi:}
\begin{minipage}{15cm}
\begin{rail}
'?' TEXT[tag] ( | '=' ( TEXT | QUOTES ))
\end{rail}
\end{minipage}

\raisebox{8pt}{doctype:}
\begin{minipage}{15cm}
\begin{rail}
'!DOCTYPE' '=' ( TEXT | QUOTES )
\end{rail}
\end{minipage}

\pagebreak

\texttt{Rules \hfill 3/3}

\vspace{5mm}

\verb|TEXT tag and attribute must conform to the XML element naming rules:|\\
\verb|    1. must start with a letter or underscore| \\
\verb|    2. cannot start with the letters xml (or XML, or Xml, etc)| \\
\verb|    3. can contain letters, digits, hyphens, underscores, and periods|

\vspace{5mm}

\verb|QUOTE content is trimmed according to these rules:|\\
\verb!    1. Any leading/ending (space|cr|tab)* nl (space|cr|tab)* is removed.! \\
\verb|    2. Incidental whitespace will always be trimmed, when there is at least one nl left.|

\vspace{5mm}

\verb|QUOTES can combine multiple QUOTE(s):|\\
\verb|    1. A suffix \ joins the next QUOTE seamlessly.|\\
\verb|    2. A suffix \n inserts a newline before joining.|

\vspace{5mm}

\verb|Joining children inside the nodes rule:|\\
\verb|    1. All children are joined seamlessly.|\\
\verb|    2. Except, two consecutive standalone QUOTEs are joined with a newline.|

\vspace{5mm}

\verb|Pretty printing introduces whitespace for div{a{span=' hi '}}|\\
\verb|    1. Default XML output will pretty print and insert newlines+indentation.|\\
\verb|       This is usually safe, use --nopp to not pretty print.|\\
\verb|       <div>|\\
\verb|           <a>|\\
\verb|               <span> hi </span>|\\
\verb|           </a>|\\
\verb|       </div>|\\
\\
\verb|    2. Default HTML output is without pretty printing since|\\
\verb|       introducing newlines+indentation in HTML requires DTD/CSS knowledge|\\
\verb|       where it is possible to safely introduce whitespace.|\\
\verb|       <div><a><span> hi </span></a></div>|\\
\\
\verb|    3. It is possible to pretty print HTML with --pp and get a best effort guess of|\\
\verb|       where it is safe to introduce newlines+indentation.|\\
\verb|       <div>|\\
\verb|           <a><span> hi </span></a></div>|\\

\end{document}
