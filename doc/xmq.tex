\documentclass[10pt,a4paper]{article}
\usepackage{rail}
\usepackage{parskip}
\usepackage{changepage}
\usepackage{geometry}
\usepackage{graphicx}
\usepackage{color}
\usepackage[absolute,overlay]{textpos}

\definecolor{Brown}{rgb}{0.86,0.38,0.0}
\definecolor{Blue}{rgb}{0.0,0.37,1.0}
\definecolor{DarkSlateBlue}{rgb}{0.28,0.24,0.55} % 483d8b
\definecolor{Green}{rgb}{0.0,0.46,0.0}
\definecolor{Red}{rgb}{0.77,0.13,0.09}
\definecolor{LightBlue}{rgb}{0.40,0.68,0.89} %67ade5
\definecolor{MediumBlue}{rgb}{0.21,0.51,0.84} %3681d5
\definecolor{LightGreen}{rgb}{0.54,0.77,0.43} %89c56d
\definecolor{Grey}{rgb}{0.5,0.5,0.5}
\definecolor{Purple}{rgb}{0.69,0.02,0.97}
\definecolor{Yellow}{rgb}{0.5,0.5,0.1}
\definecolor{Cyan}{rgb}{0.3,0.7,0.7}

\newcommand{\xmqCom}[1]{{\color{Cyan}#1}}
\newcommand{\xmqName}[1]{{\color{Brown}#1}}
\newcommand{\xmqKey}[1]{{\color{Blue}#1}}
\newcommand{\xmqAttrKey}[1]{{\color{Grey}#1}}
\newcommand{\xmqAttrVal}[1]{{\color{Green}#1}}
\newcommand{\xmqEq}[0]{{\color{DarkSlateBlue}=}}
\newcommand{\xmqQ}[0]{{\color{Purple}'}}
\newcommand{\xmqLB}[0]{{\color{DarkSlateBlue}\{}}
\newcommand{\xmqRB}[0]{{\color{DarkSlateBlue}\}}}
\newcommand{\xmqLP}[0]{{\color{Purple}(}}
\newcommand{\xmqRP}[0]{{\color{Purple})}}
\newcommand{\xmqE}[1]{{\color{Purple}\&#1}}
\newcommand{\s}[0]{\mbox{~}}
\newcommand{\xmqs}[0]{\s\s\s\s}
\newcommand{\POS}[3]{\raisebox{#1}[0mm][0mm]{\makebox[0mm][l]{\rule{#2}{0mm}#3}}}
\setlength{\unitlength}{1mm}

\makeatletter
\newcommand*{\shifttext}[2]{%
  \settowidth{\@tempdima}{#2}%
  \raisebox{0pt}[0pt][0pt]{%
  \makebox[\@tempdima]{\hspace*{#1}#2}}%
}
\makeatother

\geometry{a4paper, layoutwidth=20cm, layoutheight=32cm}

\newcommand{\shiftleft}[2]{\makebox[0pt][r]{\makebox[#1][l]{#2}}}
\newcommand{\shiftright}[2]{\makebox[#1][r]{\makebox[0pt][l]{#2}}}

\railoptions{-t}
\relax

\pagestyle{empty}

\railalias{LBRACE}{\{}
\railalias{RBRACE}{\}}
\railalias{BA}{\textbackslash}
\railalias{AMP}{{\tt\&}}
\railalias{DQUOTE}{{\tt\"}}
\railalias{VALUE}{{\tt value}}

\railterm{BA}
\railterm{AMP}
\railterm{LBRACE}
\railterm{RBRACE}
\railterm{VALUE}

\begin{document}

{\color{Red}\texttt{Specification for XMQ by Fredrik Öhrström 2023-10-07 \texttt{oehrstroem@gmail.com}}}

\texttt{Input must be valid UTF8 (0x9 | 0xa | 0xd | [20-d7ff] | [e000-fffd] | [10000-10ffff]) \\
  CRLF pairs (0xd 0xa) and standalone CR (0xd) are treated as LF (0xa) when parsing.}

\shifttext{-20mm}{\raisebox{-10cm}{\smash{\rotatebox{90}{\rule{50mm}{0.5pt}\ \texttt{LEXER}\ \rule{40mm}{.5pt}}}}}
\hbox{
  \raisebox{-4pt}{\hspace{-1em}\texttt{WS:}}
  \begin{minipage}{5cm}
    \begin{rail}
      '0x9 0xa 0xd 0x20'
    \end{rail}
  \end{minipage}
  \ \ \ \ \
  \raisebox{-4pt}{\hspace{-1em}\texttt{UWS:}}
  \begin{minipage}{5cm}
    \begin{rail}
      'any unicode whitespace'
    \end{rail}
  \end{minipage}
}

\raisebox{32pt}{\texttt{QUOTE:}}
\begin{minipage}{15cm}
\begin{rail}
  "''"
  |"'" "UTF8 with no '" "'"
  | "'''" "UTF8 max 2 consec '" "'''"
  | "'"[$\times$ n] "UTF8 max n-1 consec '" "'"[$\times$ n]
\end{rail}
\end{minipage}

\raisebox{-4pt}{\texttt{ENTITY:}}
\begin{minipage}{15cm}
\begin{rail}
   AMP "TEXT:entity" ';'
\end{rail}
\end{minipage}

\raisebox{32pt}{\texttt{COMMENT:}}
\begin{minipage}{15cm}
\begin{rail}
  "/"[$\times$ n] '*' "UTF8" (('*' "/"[$\times$ n] '*' "UTF8")*) '*' "/"[$\times$ n]
  | '//' "UTF8" "0xa"
\end{rail}
\end{minipage}

\raisebox{-4pt}{\texttt{TEXT:}}
\begin{minipage}{15cm}
\begin{rail}
"UTF8 excluding UWS and ' '\!\!' ( ) \{ \} must not start with = \& // /*"
\end{rail}
\end{minipage}

\shifttext{-20mm}{\raisebox{-10.2cm}{\smash{\rotatebox{90}{\rule{43mm}{0.5pt}\ \texttt{PARSER}\ \rule{43mm}{.5pt}}}}}

\vspace{-10pt}
\begin{minipage}{6cm}
  \raisebox{9pt}{\texttt{xmq:}}
  \begin{minipage}{5cm}
    \begin{rail}
      "node"+
    \end{rail}
  \end{minipage}
\end{minipage}
\raisebox{8pt}{\texttt{attr:}}
\begin{minipage}{15cm}
  \begin{rail}
    "TEXT:key" ( | '=' VALUE )
  \end{rail}
\end{minipage}

\vspace{-8pt}
\begin{minipage}{7cm}
  \raisebox{21pt}{\texttt{node:}}
  \begin{minipage}{6cm}
    \begin{rail}
      "COMMENT" | "qe" | "element"
    \end{rail}
  \end{minipage}

  \begin{minipage}{7cm}
    \raisebox{8pt}{\texttt{qe:}}
    \begin{minipage}{6cm}
      \begin{rail}
        "QUOTE" | "ENTITY"
      \end{rail}
    \end{minipage}
  \end{minipage}
\end{minipage}
\begin{minipage}{7cm}
  \begin{minipage}{7cm}
    \raisebox{44pt}{\texttt{value:}}
    \begin{minipage}{6cm}
      \begin{rail}
        "TEXT" | "qe" | '(' ')' | '(' ( "qe"+) ')'
      \end{rail}
    \end{minipage}
  \end{minipage}
\end{minipage}

\raisebox{44pt}{\texttt{element:}}
\begin{minipage}{15cm}
  \begin{rail}
    "TEXT:namekey" ( | '(' ')' | '(' ( "attr"+) ')' ) ( | '=' VALUE | LBRACE RBRACE | LBRACE ( "node" + ) RBRACE)
  \end{rail}
\end{minipage}

\pagebreak

\shifttext{-20mm}{\raisebox{-9.5cm}{\smash{\rotatebox{90}{\rule{40mm}{0.5pt}\ \texttt{RULES}\ \rule{40mm}{.5pt}}}}}

\verb|TEXT:namekey TEXT:key and TEXT:entity| \\
\verb|r1. must start with a letter or underscore| \\
\verb|r2. cannot start with the letters xml (or XML, or Xml, etc)| \\
\verb|r3. can contain letters, digits, hyphens, underscores, periods and a single colon.|

\verb|TEXT:namekey is either a name or a key.| \\
\verb|r4. it is a key if '=' follows immediately (ie no attributes), otherwise it is a name.|

\verb|Two permitted exceptions to TEXT:namekey rule r1 and r3.| \\
\verb|r5. ?pi-nodes and !DOCTYPE|

\verb|If quoted content contains at least one newline then:|\\
\verb|r6. All leading and ending spaces/tabs/newlines are removed.| \\
\verb|r7. Spaces/tabs before a newline are removed.| \\
\verb|r8. Incidental indentation is removed.|\\
\verb|r9. The indent to be removed is the minimum source code indentation for text within|\\
\verb|    the block where empty lines are ignored.|\\
\verb|r10. The first line is prefixed with spaces if the following lines have lesser|\\
\verb|     source code indentation.|\\
\verb||\\
\verb|A quote with only spaces and at least one newline is equivalent to the empty string.|
\texttt{The expression \xmqKey{age}\xmqEq 123    is shorthand for   \xmqName{age}\xmqLB\xmqQ 123\xmqQ\xmqRB}\\
\texttt{Use \xmqLP ...\xmqRP\ for attributes with newlines and explicit leading/ending spaces.} \\
\texttt{Inside a comment \xmqCom{*/*} means a newline, used for compact xmq without newlines.}\\

\shifttext{-20mm}{\raisebox{-6.7cm}{\smash{\rotatebox{90}{\rule{25mm}{0.5pt}\ \texttt{EXAMPLES}\ \rule{25mm}{.5pt}}}}}

\hbox{
\begin{minipage}{6cm}
\texttt{\xmqName{car} \xmqLB\\
\xmqs\xmqCom{// An example structure.}\\
\xmqs\xmqKey{regnr} \xmqEq\ \xmqQ ABC 123\xmqQ\\
\xmqs\xmqKey{color} \xmqEq\ red\\
\xmqs\xmqKey{img}\ \ \ \xmqEq\ /www/y.png\\
\xmqs\xmqKey{tag}\ \ \ \xmqEq\ <car>\\
\xmqRB
}
\end{minipage}

\rule{2cm}{0cm}

\begin{minipage}{6cm}
\verb|<car>|\\
\verb|  <!-- An example structure. -->|\\
\verb|  <regnr>ABC 123</regnr>|\\
\verb|  <color>red</color>|\\
\verb|  <img>/www/y.png</img>|\\
\verb|  <tag>&lt;car&gt;</tag>|\\
\verb|</car>|
\end{minipage}
}

\vspace{5mm}

\hbox{
\begin{minipage}{6cm}
\texttt{\xmqName{div}(\xmqAttrKey{id} \xmqEq\ \xmqAttrVal{32}) \{\\
\xmqs\xmqKey{h1}\ \xmqEq\ Welcome!\\
\xmqs\xmqQ Rest here weary\\
\xmqs \ traveller:\xmqQ\\
\xmqs \xmqName{a}(\xmqAttrKey{href} \xmqEq\ \xmqAttrVal{https://a.b.c}) \{\\
\xmqs\xmqs \xmqName{img}(\xmqAttrKey{url} \xmqEq\ \xmqAttrVal{/img/i.png})\\
\xmqs\xmqs\xmqQ Click here!\xmqQ\\
\xmqs\}\\
\}\\
}
\end{minipage}

\rule{2cm}{0cm}

\begin{minipage}{6cm}
\verb|<div id="32">|\\
\verb|  <h1>Welcome!</h1>|\\
\verb|    Rest here weary|\\
\verb|traveller:<a href="https://a.b.c">|$\hookleftarrow$\\
$\hookrightarrow$\verb|<img url="/img/i.png">Click here!</a>|\\
\verb|</div>| \\
\verb||\\
\verb||\\
\verb||
\end{minipage}
\POS{-12mm}{-47mm}{\begin{picture}(10,10)\color{Grey}\put(0,0){\vector(-1,1){11}}\put(0,0){\vector(1,4){4}}\put(0,0){\vector(3,1){30}}\put(-40,-3){\texttt{\small Html cannot be pretty printed with newlines here,}}\put(-40,-7){\texttt{\small whereas xmq can be pretty printed without introducing whitespace.}}\end{picture}}%
}

\shifttext{-20mm}{\raisebox{-3.7cm}{\smash{\rotatebox{90}{\rule{10mm}{0.5pt}\ \texttt{CORNERS}\ \rule{10mm}{.5pt}}}}}

\texttt{Explicit spaces:  \xmqKey{abc} \xmqEq\ \xmqQ \ \ \ \ \xmqQ \ \ \ \  \xmqName{abc} \xmqLB \xmqQ\ \ \ \ \xmqQ \xmqRB \\
A single newline: \xmqKey{abc} \xmqEq\ \xmqE{\#10;}\ \ \ \ \ \ \ \xmqName{abc} \xmqLB\ \xmqE{\#10;} \xmqRB \\
Spaces surrounding newline: \xmqKey{abc} \xmqEq\ \xmqLP\xmqQ\ \xmqQ\ \xmqE{\#10;} \xmqQ\ \xmqQ \xmqRP\ \ \ \xmqName{abc} \xmqLB \xmqQ\ \xmqQ\ \xmqE{\#10;} \xmqQ\ \xmqQ \xmqRB \\
Quote starting with quote characters: \xmqLP\ \xmqE{\#39;} \xmqQ quoted quote\xmqQ\ \xmqE{\#39;} \xmqRP\ or: \\
\xmqQ\xmqQ\xmqQ\\
\ \ 'quoted quote'\\
\xmqQ\xmqQ\xmqQ \\
When highlighting: \xmqKey{p} \xmqEq\ 123 is a key value and \xmqName{p}(\xmqKey{x}) \xmqEq\ 123 is a name value.
}

\end{document}
